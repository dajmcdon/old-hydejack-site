\documentclass[10pt]{article}
\usepackage{calc}

\reversemarginpar

% 
% PAPER SIZE, PAGE NUMBER, AND DOCUMENT LAYOUT NOTES:
% 
% The next \usepackage line changes the layout for CV style section
% headings as marginal notes. It also sets up the paper size as either
% letter or A4. By default, letter was used. If A4 paper is desired,
% comment out the letterpaper lines and uncomment the a4paper lines.
% 
% As you can see, the margin widths and section title widths can be
% easily adjusted.
% 
% ALSO: Notice that the includefoot option can be commented OUT in order
% to put the PAGE NUMBER *IN* the bottom margin. This will make the
% effective text area larger.
% 
% IF YOU WISH TO REMOVE THE ``of LASTPAGE'' next to each page number,
% see the note about the +LP and -LP lines below. Comment out the +LP
% and uncomment the -LP.
% 
% IF YOU WISH TO REMOVE PAGE NUMBERS, be sure that the includefoot line
% is uncommented and ALSO uncomment the \pagestyle{empty} a few lines
% below.
% 

%% Use these lines for letter-sized paper
\usepackage[paper=letterpaper,
% includefoot, % Uncomment to put page number above margin
marginparwidth=1.2in,     % Length of section titles
marginparsep=.05in,       % Space between titles and text
margin=1in,               % 1 inch margins
includemp]{geometry}

%% Use these lines for A4-sized paper
% \usepackage[paper=a4paper,
%            % includefoot, % Uncomment to put page number above margin
% marginparwidth=30.5mm,    % Length of section titles
% marginparsep=1.5mm,       % Space between titles and text
% margin=25mm,              % 25mm margins
% includemp]{geometry}

\setlength{\parindent}{0in}
\usepackage{paralist}

%% Reference the last page in the page number
% 
% NOTE: comment the +LP line and uncomment the -LP line to have page
% numbers without the ``of ##'' last page reference)
% 
% NOTE: uncomment the \pagestyle{empty} line to get rid of all page
% numbers (make sure includefoot is commented out above)
% 
\usepackage{fancyhdr,lastpage}
\pagestyle{fancy}
% \pagestyle{empty}      % Uncomment this to get rid of page numbers
\fancyhf{}\renewcommand{\headrulewidth}{0pt}
\fancyfootoffset{\marginparsep+\marginparwidth}
\newlength{\footpageshift}
\setlength{\footpageshift}
{0.5\textwidth+0.5\marginparsep+0.5\marginparwidth-2in}
\lfoot{\hspace{\footpageshift}%
  \parbox{4in}{\, \hfill %
    McDonald --- \arabic{page} of \protect\pageref*{LastPage} % +LP
    % \arabic{page}                               % -LP
    \hfill \,}}

% Finally, give us PDF bookmarks
\usepackage{color,hyperref}
\definecolor{darkblue}{rgb}{0.0,0.0,0.4}
\hypersetup{colorlinks,breaklinks,
  linkcolor=darkblue,urlcolor=darkblue,
  anchorcolor=darkblue,citecolor=darkblue}


%%%%%%%%%%%%%%%%%%%%%%%% End Document Setup %%%%%%%%%%%%%%%%%%%%%%%%%%%%


%%%%%%%%%%%%%%%%%%%%%%%%%%% Helper Commands %%%%%%%%%%%%%%%%%%%%%%%%%%%%

% The title (name) with a horizontal rule under it
% 
% Usage: \makeheading{name}
% 
% Place at top of document. It should be the first thing.
\newcommand{\makeheading}[1]%
{\hspace*{-\marginparsep minus \marginparwidth}%
  \begin{minipage}[t]{\textwidth+\marginparwidth+\marginparsep}%
    {\large \bfseries #1}\\[-0.15\baselineskip]%
    \rule{\columnwidth}{1pt}%
  \end{minipage}}

\newcommand{\email}[1]{\href{mailto:#1}{#1}}
% The section headings
% 
% Usage: \section{section name}
% 
% Follow this section IMMEDIATELY with the first line of the section
% text. Do not put whitespace in between. That is, do this:
% 
% \section{My Information}
% Here is my information.
% 
% and NOT this:
% 
% \section{My Information}
% 
% Here is my information.
% 
% Otherwise the top of the section header will not line up with the top
% of the section. Of course, using a single comment character (%) on
% empty lines allows for the function of the first example with the
% readability of the second example.
\renewcommand{\section}[2]%
{\pagebreak[2]\vspace{1.3\baselineskip}%
  \phantomsection\addcontentsline{toc}{section}{#1}%
  \hspace{0in}%
  \marginpar{
    \raggedright \scshape #1}#2}

% An itemize-style list with lots of space between items
\newenvironment{outerlist}[1][\enskip\textbullet]%
{\begin{itemize}[#1]}{\end{itemize}%
  \vspace{-.6\baselineskip}}

% An environment IDENTICAL to outerlist that has better pre-list spacing
% when used as the first thing in a \section 
\newenvironment{outerlist1}[1][\enskip\textbullet]%
{\vspace{-\baselineskip}\begin{list}{#1}{%
      \setlength{\partopsep}{0pt}%
      \setlength{\topsep}{0pt}}}
  {\end{list}\vspace{-.6\baselineskip}}

% An itemize-style list with little space between items
\newenvironment{innerlist}[1][\enskip\textbullet]%
{\begin{compactitem}[#1]}{\end{compactitem}}

% An environment IDENTICAL to outerlist that has better pre-list spacing
% when used as the first thing in a \section 
\newenvironment{innerlist1}[1][\enskip\textbullet]%
{\vspace{-\baselineskip}\begin{compactitem}[#1]}{\end{compactitem}}

% To add some paragraph space between lines.
% This also tells LaTeX to preferably break a page on one of these gaps
% if there is a needed pagebreak nearby.
\newcommand{\blankline}{\quad\pagebreak[2]}

%%%%%%%%%%%%%%%%%%%%%%%% End Helper Commands %%%%%%%%%%%%%%%%%%%%%%%%%%%

%%%%%%%%%%%%%%%%%%%%%%%%% Begin CV Document %%%%%%%%%%%%%%%%%%%%%%%%%%%%

\begin{document}
\makeheading{Daniel J.~McDonald}

\section{Contact Information}
% DON'T LEAVE BLANK LINES AFTER SECTION HEADINGS!!!
% NOTE: Mind where the & separators and \\ breaks are in the following
% table.
% 
% ALSO: \rcollength is the width of the right column of the table 
% (adjust it to your liking; default is 1.85in).
% 
\newlength{\rcollength}\setlength{\rcollength}{2.25in}%
\newlength{\rrcollength}\setlength{\rrcollength}{1.75in}%
% 
\begin{tabular}[t]{@{}p{\textwidth-2.25in}p{2.25in}}
  \href{http://www.stat.cmu.edu/}%
  {Department of Statistics} & \textit{Phone:} (xxx) xxx--xxxx \\
  \href{http://www.cmu.edu/}{Carnegie Mellon University}&
  \textit{E-mail:}
  \email{danielmc@cmu.edu}\\ 
  Baker Hall 132           & \\
  Pittsburgh, PA 15213   & \\
  \url{http://www.stat.cmu.edu/~danielmc/}
\end{tabular}

\section{Education}
% 
\href{http://www.cmu.edu/}{\textbf{Carnegie Mellon University}}, 
Pittsburgh, Pennsylvania USA
\begin{outerlist}[]


\item Ph.D Candidate, 
  \href{http://www.stat.cmu.edu/}
  {Statistics} 
  (expected graduation date: May 2012)
\begin{innerlist}[]
  \item \textbf{Dissertation:} ``Generalization error bounds for
    state-space models with an application to economic forecasting''
  \item \textbf{Dissertation advisors:}
    \href{http://www.stat.cmu.edu/~cshalizi} {Cosma Shalizi} and
    \href{http://www.stat.cmu.edu/~cshalizi} {Mark Schervish} 
\end{innerlist}


\item M.S., 
  \href{http://www.stat.cmu.edu/}
  {Statistics}, May 2008
\end{outerlist}

\blankline

\href{http://www.iub.edu/}{\textbf{Indiana University}}, Bloomington,
Indiana USA 
\begin{outerlist}[]
\item Bachelor of Arts, 
  \href{http://www.indiana.edu/~econweb/}
  {Economics}, May 2006
  \begin{innerlist}
  \item \emph{Magna cum laude}
  \item Member of Phi Beta Kappa
  \item Member of the Indiana University, Bloomington Honors College
  \end{innerlist}
\item Bachelor of Science in
  \href{http://www.music.indiana.edu/}{Music} (Cello) and an Outside
  Field (\href{http://www.math.indiana.edu/}{Mathematics}), May 2006 
\end{outerlist}

\section{Peer reviewed publications}
\textsc{McDonald, D.J., Shalizi, C.R., and Schervish, M.}, (2011),
``Estimating $\beta$-mixing coefficients,'' in \emph{Proceedings of the
Fourteenth International Conference on Artificial Intelligence and
Statistics}, eds. G.~Gordon, D.~Dunson, and M.~Dud{\'\i}k,
\href{http://jmlr.csail.mit.edu/proceedings/papers/v15/mcdonald11a.html}
{JMLR W\&CP}, \textbf{15}, 516--524,
\href{http://arxiv.org/abs/1103.0941}{arXiv:1103.0941}.\\ 

\textsc{McDonald, D.J.~and Thornton, D.L.}, (2007), ``A Primer on the
mortgage market and mortgage finance,'' Federal Reserve Bank of
St. Louis
\href{http://research.stlouisfed.org/publications/review/article/6257}{\emph{Review}},
\textbf{90}, 31--46.

\section{Papers under review}
\textsc{McDonald, D.J., Shalizi, C.R., and Schervish, M.}, (2011),
``Estimated VC dimension for risk bounds,'' submitted for
publication. \href{http://arxiv.org/abs/1111.3404}{arXiv:1111.3404}.\\


\section{Papers in progress}
\textsc{McDonald, D.J., Shalizi, C.R., and Schervish, M.},
``Macroeconomic forecasting: model evaluation and selection using
nonparametric risk bounds.''\\


\section{Grants}
\href{http://ineteconomics.org/grants/model-complexity-and-prediction-error-macroeconomic-forecasting} 
{Model Complexity and Prediction Error in Macroeconomic Forecasting}
(with C.~Shalizi and M.~Schervish),
\href{http://ineteconomics.org}{\em Institute for New Economic
  Thinking}

\section{Conference and Seminar Presentations}
\begin{innerlist1}[-]
\item The Classification Society Annual Meeting (2011)
\item Statistical Machine Learning Group, CMU (2010, 2011)
\item $14^{th}$ International Conference on Artificial Intelligence
  and Statistics (2011)
\item American Statistical Association, Pittsburgh Chapter Annual
  Meeting (2011)
\item Joint Statistical Meetings (2010)
\end{innerlist1}

\section{Work Experience}
\href{http://www.cmu.edu}{\textbf{Carnegie Mellon University}}, 
Pittsburgh, Pennsylvania
\begin{outerlist}[]
\item \textit{Instructor}
  \hfill \textit{Summer 2009 and 2010}
  \begin{innerlist}
  \item Taught 36-226 ``Introduction to probability and statistics
    II''. Created lectures, assignments, and exams.
  \end{innerlist}
\item \textit{Teaching Assistant}% 
  \hfill \textit{September 2007 to present}
  \begin{innerlist}
  \item Implemented statistical arbitrage techniques (pairs trading,
    relative value and momentum strategies, volatility arbitrage) and
    evaluated them on market data, in preparation of material for
    Statistical Arbitrage course.  
  \item Teaching Assistant for ``Statistical Reasoning and Practice'' and
    ``Engineering Statistics and Quality Control,'' (undergraduate);
    ``Advanced Statistical Theory'' (PhD); ``Probability,''
    ``Statistical Inference,'' ``Linear Models,'' ``Statistical
    Arbitrage,'' and ``Topics in Quantitative Finance'' (MS in
    Computational Finance, Tepper School of Business).
  \end{innerlist}
\end{outerlist}

\blankline


\href{http://www.stlouisfed.org/}{\textbf{Federal Reserve Bank of
  St. Louis}}, St. Louis, Missouri
\begin{outerlist}[]
\item \textit{Research Associate} \hfill \textit{July 2006 to July 2007}
  \begin{innerlist}
  \item Provide research support for 2 economists' research projects
    intended for publication in peer-reviewed journals:  write
    statistical routines, provide econometric analysis, collect and
    process data, create figures, tables and datasets, proofread
    manuscripts. 
  \item Significant experience writing econometric/statistical routines
    is SAS, Eviews, and Matlab 
  \end{innerlist}
\end{outerlist}

\section{Professional Service}
Referee: \emph{Journal of Business and Economic Statistics},
\emph{IEEE Transactions on Information Theory}\\


Memberships: \emph{American Statistical Association},
\emph{Institute of Mathematical Statistics},
\emph{Bernoulli Society},
\emph{American Economic Association},
\emph{Econometric Society}

\section{Computing Skills}
\textit{Statistical Packages:} R/S-Plus, Matlab, SAS\\
\textit{Programming Languages:} C/C++, \LaTeX, Python, OpenCL\\
\textit{Operating Systems:} MacOS, Linux, Windows\\


\section{References} 
References available upon request

\end{document}

%%%%%%%%%%%%%%%%%%%%%%%%%% End CV Document %%%%%%%%%%%%%%%%%%%%%%%%%%%%%
